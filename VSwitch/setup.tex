\section{Hardware}

You will be given a 4-port Ethernet USB adapter that you can use to
add four physical ports to any PC.  If you use the laboratory PCs, be
aware that some of the USB ports provide insufficient power. Which
ones work is inconsistent even across identical PCs and often even the
adjacent USB port works even though it looks identical!

You are not expected to write a driver to interact directly with the
Ethernet USB adapter.  Instead, you will use the provided
{\tt network-driver} which can already provide you with raw access
to any Ethernet interface (incl. WLAN).

\subsection{Alternative setup with virtual machines}


Clone the Git repository at
\url{https://gitlab.ti.bfh.ch/demos/vlab} and follow the provided
instructions.

\section{The {\tt network-driver}}

To access the hardware, your final program should be {\em executed}
by the {\tt network-driver}.  For this, you call
\begin{verbatim}
$ network-driver IFC1 ... IFCn - ./switch ARGS
\end{verbatim}
where ``IFC1 ... IFCn'' is the list of interface names that you want
{\tt network-driver} to support (i.e. ``lan0'', ``lan1'') and ``PROG''
is the name of your binary and ``ARGS'' are the command-line arguments
to ``switch''.  Note the ``-'' (single minus) between the last
interface name and ``switch''.  Also, ``./switch'' must be given with
its path (i.e. ``./switch'' for the current working directory) or be
located in a directory that is given in the ``PATH'' environment
variable.

Once you start {\tt switch} like this, you can read Ethernet frames
and end-user commands from ``stdin'' and write Ethernet
frames (and end-user output) to ``stdout''.

Note that you must follow the {\tt network-driver}'s particular
format for inter-process communication when reading and writing.
You will {\bf not} be communicating directly with the console!


\subsection{Build the driver}

To compile the code, run:
\begin{verbatim}
# This requires gcc
$ make
# Creating network interfaces requires 'root' rights
$ sudo chmod +s network-driver
# Try it out:
$ ./network-driver eth0 - ./parser
\end{verbatim}
Press CTRL-C to stop the {\tt network-driver} and {\tt parser}.


\subsection{Understanding the driver}

The output of the driver is always in binary and generally in network
byte order.  You can use a tool like {\tt hexer} to make the output
slightly more readable.

The driver will always output a series of messages starting with
a {\tt struct GLAB\_MessageHeader} that includes a type and a size.

When the driver starts, it first writes a control message (of type 0)
with payload that includes 6 bytes for each of the local interface's
MAC addresses to your {\tt stdin}.  Henceforce, messages received
of type 0 will be single lines of command-line input (including the
'\\n'-terminator, but excluding the 0-terminator of C) as typed in
by the user.

Furthermore, the driver will output a {\tt struct GLAB\_MessageHeader}
for each frame received.  The {\tt struct GLAB\_MessageHeader} will be
followed by the actual network frame, starting with the Ethernet frame
excluding preamble, delimiter and FCS.  The {\tt struct
  GLAB\_MessageHeader} includes the total length of the subsequent
frame (encoded in network byte order, the size includes the {\tt
  struct GLAB\_MessageHeader}).  The fixed message type identifies the
number of the network interface, counting from one (also in network
byte order).

In addition to writing received frames to your {\tt stdin}, the driver
also tries to read from your {\tt stdout}.  Applications must send the
same message format to {\tt stdout} that the driver sends them on {\tt
  stdin}.  The driver does {\bf not} check that the source MAC is set
correctly!

To write to the console's {\tt stdout}, use a message type of 0.
You may directly write to {\tt stderr} for error messages.
