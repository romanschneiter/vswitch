\documentclass{article}
\usepackage{url}
\usepackage{upquote}

\title{BTI 3021: Networking Project - Sprint 3}

\author{Christian Grothoff}
\date{}

\begin{document}
\maketitle

\section{Introduction}

For this sprint you will write an IP router, building upon the
results from your previous sprint (ARP). You will most likely
want to copy large parts of your ARP code into the router logic.

While the driver and skeleton you are given is written in C, you may
again use {\em any} language of your choice for the implementation (as
long as you extend the {\tt Makefile} with adequate build rules).
However, if you choose a different language, be prepared to write
additional boilerplate yourselves.

How an IP router works is expected to be understood from the
networking class. If not, you can find plenty of documentation and
specifications on the Internet.

The basic setup is the same as in the first two sprints.

\subsection{Deliverables}

There will be two main deliverables for the sprint:

\begin{description}
\item[Implementation] You must implement an IPv4 router. Your
  implementation must answer to IP packets, and route them.
  For this, you are to extend the {\tt router.c} template provided
  (or write the entire logic from scratch in another language).
\item[Testing] You must implement and submit your own {\bf test cases}
  by {\em pretending} to be the network driver (see below) and sending
  IP packets or command-line inputs to your program and verifying that it
  outputs the correct frames. Additionally, you should perform
  {\em interoperability} tests against existing
  implementations (i.e. other notebooks from your team to ensure that
  your IP router implementation integrates correctly with other
  implementations).
\end{description}


\subsection{Functionality}

Implement {\tt router} which routes IPv4 packets:
\begin{enumerate}
\item Populate your routing table from the network interface configuration
  given on the command-line using the same syntax as with the {\tt arp}
  program.
\item Use the ARP logic to resolve the target MAC address.   Simply drop the IP
  packets for destinations where the next hop's MAC address has not yet been
  learned, and issue the ARP request to obtain the destination’s MAC instead
  (once per dropped IP packet).
\item Make sure to decrement the TTL field and recompute the CRC.
  % add link to logic implementing CRC?
\item Generate ICMP messages for ``no route to network'' (ICMP
    Type 3, Code 0) and ``TTL exceeded'' (ICMP Type 11, Code 0),
\item Support the syntax {\tt IFC[RO]=MTU} where {\tt MTU} is the
  MTU for IFC.  Example: {\tt eth0=1500}.  Implement and test IPv4 fragmentation
  (including {\em do not fragment}-flag support), including sending
  ICMP  (ICMP Type 3, Code 4).
\item Support dynamic updates to the routing table via {\tt stdin}.
  Base your commands on the {\tt ip route} tool.  For example,
  ``route list'' should output the routing table, and
  ``route add 1.2.0.0/16 via 192.168.0.1 dev eth0'' should add
  a route to {\tt 1.2.0.0/16} via the next hop {\tt 192.168.0.1}
  which should be reachable via {\tt eth0}.  Implement at least
  the {\tt route list}, {\tt route add} and {\tt route del} commands.
  The interface-specific (connected local network) routes that
  are added upon startup from the command-line must not need to be
  {\tt del}etable.
\end{enumerate}

The output of your routing table should have the following format:
\begin{verbatim}
192.168.0.0/255.255.0.0 -> 1.2.3.4 (eth0)
\end{verbatim}
Use 0.0.0.0 if there is no next hop (the target host is in the connected
LAN on the specified interface).  You may print the routing table
in any order. Do include locally connected networks.

Routing table entries for locally connected networks MUST NOT be
configured explictly (via ``route add``) but must be automatically
created when your router starts (from the command-line arguments). You
do not have to support removal of those routing table entries.


Note that your implementation must realize following functions of a
router:

\begin{itemize}
\item Basic IP handling (TTL, ICMP, Checksum) % IP TTL decremented, ICMP? Checksum?
\item Forwarding and routing % IP packets flow? Mac updated?
\item Address resultion and caching % ARP cache?
\item IP fragmentation % Use eth3 for testing
\end{itemize}


\subsection{Testing}

Note that in addition to automated tests similar to the {\tt
  public-test-router} you {\bf should} probably do integration tests
in a real network (evaluating manually with ping, wireshark, etc.).
This will help you find problems in your implementation and ensure
that your understanding of the network protocols is correct.


\section{Grading}

\section{Grading}

{\bf All deliverables must be submitted to Git} (master branch) by the
submission deadline announced on Moodle.

You are expected to work in a team of {\bf four students}.  If needed,
the course coordinator may permit the creation of a team of 5
students.  Each team is responsible for dividing up the work and
coordinating as needed.

\subsection{BTI 3021}
You can earn 14, 14 and 22 points in the three networking sprints, for
a total of 50 points.  A canonical team will need {\bf 37 points} to
pass the networking component of BTI 3021.

\subsection{BTI 3022}
You can earn 14 and 22 points in the three networking sprints, for
a total of 36 points.  A canonical team will need {\bf 27 points} to
pass the networking component of BTI 3022.

\subsection{BTI 302x repeaters}

You can earn 24 points in a single sprint.
A team of two will need all {\bf 24 points}
pass the networking component of BTI 302x
when repeating.
A team of one will need {\bf 20 points} to
pass the networking component of BTI 302x
when repeating.

\subsection{Smaller teams for non-repeaters}

If your team is {\bf for good reasons} smaller than four students, the
passing threshold will be lowered by {\bf 2 points} per ``missing''
student per sprint.  So a student going alone in BTI 3021 would still
need {\bf 19 points} to pass.

Teams may {\bf request} changes to team membership at the end of a
sprint, but must provide a {\bf justification} to the course
coordinator, who may approve or decline the request.


\subsection{Router grading}

You get points for each of the key deliverables:
\begin{center}
\begin{tabular}{l|r}
Correct implementation                    & 14 \\ \hline
Comprehensive test cases                  &  8 \\ \hline \hline
Total                                     & 22
\end{tabular}
\end{center}


\subsubsection{Correct implementation}
\begin{itemize}
\item 14 points for passing test cases
\end{itemize}

\subsubsection{Comprehensive test cases}
\begin{itemize}
\item 0 points if public reference implementation (see Section~\ref{sec:binaries})
      fails test cases, {\bf otherwise}
\item 8 points for failing buggy implementations (see Section~\ref{sec:binaries})
\end{itemize}

If you believe you have found a bug in the provided reference
implementation, you are encouraged to discuss it with the
instructor. If you have found an actual bug in the reference
implementation (that is within the scope of the assignment), you will
be awarded a {\bf bonus point} per acknowledged bug.


\section{Provided code}


You are given a few C snippets as starting points. However, these
mostly serve to {\em illustrate} how to process the output from the
driver. You are completely free to implement your application in {\em
  any} programming language.  Note that each file includes about 20
LOC of a licensing statement, so the functions provided should not
provide a significant advantage for implementations in C.

\begin{description}
\item[parser.c]{This file includes a simple starting point for
  a wireshark-like frame inspection code.
  It mostly shows how the frames are received and
  a bit how to use the other C files. (82 LOC)}
\item[glab.h]{A struct defining a MAC Address and a few common C includes. (90 LOC)}
\item[print.c]{This file shows how to wrap messages to print them
  via the driver. (112 LOC)}
\item[loop.c]{This could be the main loop of your application. The code includes
  some basic logic that looks at each frame, decides whether it is the MACs,
  control or an Ethernet frame and then calls the respective function. (93 LOC)}
\item[crc.c]{An implementation of checksum algorithms. (194 LOC)}
\end{description}

If you are using another programming language, you are free to re-use
an existing CRC implementation in that language.


The main file for the exercise is {\tt router.c}. In this file, you
should implement a program {\tt router} which implements an IPv4 router.


\section{Provided binaries} \label{sec:binaries}

You are provided with several binaries:

\begin{description}
\item[reference-test-router] A public test case, run using ``./reference-test-router ./router''
  to test your router. Returns 0 on success.
\item[reference-router] Reference implementation of the ``router''.
\item[bug1-router] Buggy implementation of a ``router``.
\item[bug2-router] Buggy implementation of a ``router``.
\item[bug3-router] Buggy implementation of a ``router``.
\item[bug4-router] Buggy implementation of a ``router``.
\end{description}

\newpage
\section{Required make targets}

You may modify the build system. However, the final build system must
have the following {\tt make} targets:

\begin{description}
\item[all] build all binaries
\item[clean] remove all compiled files
\item[router] build your ``router`` binary from source; the binary MUST end up in the top-level directory of your build tree.
\item[test-router] build your ``test-router`` program from source; the program MUST end up in the top-level directory of your build tree.
\item[check-router] Run ``test-router`` against the ``router'' binary.
\end{description}

For grading, we will basically run commands like:
\begin{verbatim}
GRADE=0
make test-router
cp bug1-router router
make check-router || GRADE=`expr $GRADE + 2`
cp bug2-router router
make check-router || GRADE=`expr $GRADE + 2`
cp bug3-router router
make check-router || GRADE=`expr $GRADE + 2`
cp bug4-router router
make check-router || GRADE=`expr $GRADE + 2`
cp refernece-router router
make check-router || GRADE=0
echo "Test grade: $GRADE"
\end{verbatim}
You must thus make sure the build system continues to create programs in the
right (top-level) location!


\end{document}
