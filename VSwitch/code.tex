
You are given a few C snippets as starting points. However, these
mostly serve to {\em illustrate} how to process the output from the
driver. You are completely free to implement your application in {\em
  any} programming language.  Note that each file includes about 20
LOC of a licensing statement, so the functions provided should not
provide a significant advantage for implementations in C.

\begin{description}
\item[parser.c]{This file includes a simple starting point for
  a wireshark-like frame inspection code.
  It mostly shows how the frames are received and
  a bit how to use the other C files. (82 LOC)}
\item[glab.h]{A struct defining a MAC Address and a few common C includes. (90 LOC)}
\item[print.c]{This file shows how to wrap messages to print them
  via the driver. (112 LOC)}
\item[loop.c]{This could be the main loop of your application. The code includes
  some basic logic that looks at each frame, decides whether it is the MACs,
  control or an Ethernet frame and then calls the respective function. (93 LOC)}
\item[crc.c]{An implementation of checksum algorithms. (194 LOC)}
\end{description}

If you are using another programming language, you are free to re-use
an existing CRC implementation in that language.
