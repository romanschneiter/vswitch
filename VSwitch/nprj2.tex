\documentclass{article}
\usepackage{url}
\usepackage{upquote}

\title{BTI 3021: Networking Project - Sprint 2}

\author{Christian Grothoff}
\date{}

\begin{document}
\maketitle

\section{Introduction}

For this sprint you will write a precursor to an IP router.  This
precursor system is to realize the ARP protocol functionality of
an IP device.

While the driver and skeleton you are given is written in C, you may
use {\em any} language of your choice for the implementation (as long
as you extend the {\tt Makefile} with adequate build rules).  However,
if you choose a different language, be prepared to write additional
boilerplate yourselves.

How the ARP protocol works is expected to be understood from the
networking class. If not, you can find plenty of documentation and
specifications on the Internet.

The basic setup is the same as in the first sprint.

\subsection{Deliverables}

There will be two main deliverables for the sprint:

\begin{description}
\item[Implementation] You must {\bf implement the ARP protocol}. Your
  implementation must answer to ARP requests, and also itself
  have the ability to issue ARP requests and to cache ARP replies.
  For this, you are to extend the {\tt arp.c} template provided
  (or write the entire logic from scratch in another language).
\item[Testing] You must implement and submit your own {\bf test cases}
  by {\em pretending} to be the network driver (see below) and sending
  ARP requests or command-line inputs to your program and verifying that it
  outputs the correct frames. Additionally, you should perform
  {\em interoperability} tests against existing
  implementations (i.e. other notebooks from your team to ensure that
  your ARP protocol implementation integrates correctly with other
  implementations).
\end{description}

All deliverables must be submitted to Git (master branch)
by the submission deadline announced on Moodle.

\subsection{Functionality}

The goal is to implement a program {\tt arp} that:
\begin{enumerate}
\item Watches for ARP queries on the Ethernet link and responds with ARP responses
  if queries are seen for your own IP address(es)
\item Provides an ARP cache so that it does not have to repeatedly
  make ARP requests to the network for MAC addresses it already knows.
\item Allows the user to trigger ARP requests via the console
  by reading ``arp`` commands with IPv4 addresses from {\tt stdin} (in human-readable format).
  The interactive command syntax should be
  ``arp {\em IP-ADDR} {\em IFNAME}'' (i.e. each line is to be prefixed with
  the letters ``arp '', followed by the IPv4 address and the name of
  the network interface).
  \begin{enumerate}
  \item
  If the {\tt IP-ADDR} is in the ARP cache, the program must immediately
  output the associated {\em MAC}:
\begin{verbatim}
28:c6:3f:1a:0a:bf
\end{verbatim}
  \item
  If the {\tt IP-ADDR} is {\bf not} in the ARP cache, the program should {\em only}
  issue the ARP query for those IPv4 addresses.
  \end{enumerate}
\item If an ARP request for one of your IP addresses
  or an ARP response destined for your system (IP and MAC)
  is received (at any time), the ARP cache must be
  updated accordingly. However, the MAC address MUST NOT be output at
  that time, even if there was an explicit command-line request for this
  address before.
\item If the user just enteres ``arp'' without
  an IP address, you should output the ARP table in the format
  ``{\em IP} -$>$ {\em MAC} ({\em IFNAME})'' with one entry per line,
  i.e.
\begin{verbatim}
10.54.25.15 -> 28:c6:3f:1a:0a:bf (eth1)
\end{verbatim}
  (note the leading ``0'' digit in {\tt 0a}).
\end{enumerate}

Your programm should be invoked with the name of the interface, the IP
address\footnote{You may support multiple IPs per network interface,
  using a comma-separated list of IPs and network masks, but this is
  not required.} for that interface and the network mask.  Example:
\begin{verbatim}
$ network-driver eth0 eth1 - \
  arp eth0[IPV4:192.168.0.1/16] eth1[IPV4:10.0.0.3/24]
\end{verbatim}
This means {\tt eth0} is to be bound to 192.168.0.1 (netmask 255.255.0.0)
and {\tt eth1} uses 10.0.0.3 (netmask 255.255.255.0).

The file {\tt arp.c} provides a starting point where the parsing of
the command-line arguments and the {\tt stdin}-interaction have been
stubbed for you.


\subsection{Testing}

Note that in addition to automated tests similar to the {\tt
  public-test-router} you {\bf should} probably do integration tests
in a real network (evaluating manually with ping, wireshark, etc.).
This will help you find problems in your implementation and ensure
that your understanding of the network protocols is correct.


\section{Grading}

\section{Grading}

{\bf All deliverables must be submitted to Git} (master branch) by the
submission deadline announced on Moodle.

You are expected to work in a team of {\bf four students}.  If needed,
the course coordinator may permit the creation of a team of 5
students.  Each team is responsible for dividing up the work and
coordinating as needed.

\subsection{BTI 3021}
You can earn 14, 14 and 22 points in the three networking sprints, for
a total of 50 points.  A canonical team will need {\bf 37 points} to
pass the networking component of BTI 3021.

\subsection{BTI 3022}
You can earn 14 and 22 points in the three networking sprints, for
a total of 36 points.  A canonical team will need {\bf 27 points} to
pass the networking component of BTI 3022.

\subsection{BTI 302x repeaters}

You can earn 24 points in a single sprint.
A team of two will need all {\bf 24 points}
pass the networking component of BTI 302x
when repeating.
A team of one will need {\bf 20 points} to
pass the networking component of BTI 302x
when repeating.

\subsection{Smaller teams for non-repeaters}

If your team is {\bf for good reasons} smaller than four students, the
passing threshold will be lowered by {\bf 2 points} per ``missing''
student per sprint.  So a student going alone in BTI 3021 would still
need {\bf 19 points} to pass.

Teams may {\bf request} changes to team membership at the end of a
sprint, but must provide a {\bf justification} to the course
coordinator, who may approve or decline the request.


\subsection{ARP grading}

You get points for each of the key deliverables:
\begin{center}
\begin{tabular}{l|r}
Correct implementation                    & 10 \\ \hline
Comprehensive test cases                  &  4 \\ \hline \hline
Total                                     & 14
\end{tabular}
\end{center}


\subsubsection{Correct implementation}
\begin{itemize}
  \item 10 points for passing test cases
\end{itemize}

\subsubsection{Comprehensive test cases}
\begin{itemize}
\item 0 points if public reference implementation (see Section~\ref{sec:binaries})
      fails test cases, {\bf otherwise}
\item 4 points for failing buggy implementations (see Section~\ref{sec:binaries})
\end{itemize}

If you believe you have found a bug in the provided reference
implementation, you are encouraged to discuss it with the
instructor. If you have found an actual bug in the reference
implementation (that is within the scope of the assignment), you will
be awarded a {\bf bonus point} per acknowledged bug.


\section{Hardware}

You will be given a 4-port Ethernet USB adapter that you can use to
add four physical ports to any PC.  If you use the laboratory PCs, be
aware that some of the USB ports provide insufficient power. Which
ones work is inconsistent even across identical PCs and often even the
adjacent USB port works even though it looks identical!

You are not expected to write a driver to interact directly with the
Ethernet USB adapter.  Instead, you will use the provided
{\tt network-driver} which can already provide you with raw access
to any Ethernet interface (incl. WLAN).

\subsection{Alternative setup with virtual machines}


Clone the Git repository at
\url{https://gitlab.ti.bfh.ch/demos/vlab} and follow the provided
instructions.

\section{The {\tt network-driver}}

To access the hardware, your final program should be {\em executed}
by the {\tt network-driver}.  For this, you call
\begin{verbatim}
$ network-driver IFC1 ... IFCn - ./switch ARGS
\end{verbatim}
where ``IFC1 ... IFCn'' is the list of interface names that you want
{\tt network-driver} to support (i.e. ``lan0'', ``lan1'') and ``PROG''
is the name of your binary and ``ARGS'' are the command-line arguments
to ``switch''.  Note the ``-'' (single minus) between the last
interface name and ``switch''.  Also, ``./switch'' must be given with
its path (i.e. ``./switch'' for the current working directory) or be
located in a directory that is given in the ``PATH'' environment
variable.

Once you start {\tt switch} like this, you can read Ethernet frames
and end-user commands from ``stdin'' and write Ethernet
frames (and end-user output) to ``stdout''.

Note that you must follow the {\tt network-driver}'s particular
format for inter-process communication when reading and writing.
You will {\bf not} be communicating directly with the console!


\subsection{Build the driver}

To compile the code, run:
\begin{verbatim}
# This requires gcc
$ make
# Creating network interfaces requires 'root' rights
$ sudo chmod +s network-driver
# Try it out:
$ ./network-driver eth0 - ./parser
\end{verbatim}
Press CTRL-C to stop the {\tt network-driver} and {\tt parser}.


\subsection{Understanding the driver}

The output of the driver is always in binary and generally in network
byte order.  You can use a tool like {\tt hexer} to make the output
slightly more readable.

The driver will always output a series of messages starting with
a {\tt struct GLAB\_MessageHeader} that includes a type and a size.

When the driver starts, it first writes a control message (of type 0)
with payload that includes 6 bytes for each of the local interface's
MAC addresses to your {\tt stdin}.  Henceforce, messages received
of type 0 will be single lines of command-line input (including the
'\\n'-terminator, but excluding the 0-terminator of C) as typed in
by the user.

Furthermore, the driver will output a {\tt struct GLAB\_MessageHeader}
for each frame received.  The {\tt struct GLAB\_MessageHeader} will be
followed by the actual network frame, starting with the Ethernet frame
excluding preamble, delimiter and FCS.  The {\tt struct
  GLAB\_MessageHeader} includes the total length of the subsequent
frame (encoded in network byte order, the size includes the {\tt
  struct GLAB\_MessageHeader}).  The fixed message type identifies the
number of the network interface, counting from one (also in network
byte order).

In addition to writing received frames to your {\tt stdin}, the driver
also tries to read from your {\tt stdout}.  Applications must send the
same message format to {\tt stdout} that the driver sends them on {\tt
  stdin}.  The driver does {\bf not} check that the source MAC is set
correctly!

To write to the console's {\tt stdout}, use a message type of 0.
You may directly write to {\tt stderr} for error messages.


\section{Provided code}


You are given a few C snippets as starting points. However, these
mostly serve to {\em illustrate} how to process the output from the
driver. You are completely free to implement your application in {\em
  any} programming language.  Note that each file includes about 20
LOC of a licensing statement, so the functions provided should not
provide a significant advantage for implementations in C.

\begin{description}
\item[parser.c]{This file includes a simple starting point for
  a wireshark-like frame inspection code.
  It mostly shows how the frames are received and
  a bit how to use the other C files. (82 LOC)}
\item[glab.h]{A struct defining a MAC Address and a few common C includes. (90 LOC)}
\item[print.c]{This file shows how to wrap messages to print them
  via the driver. (112 LOC)}
\item[loop.c]{This could be the main loop of your application. The code includes
  some basic logic that looks at each frame, decides whether it is the MACs,
  control or an Ethernet frame and then calls the respective function. (93 LOC)}
\item[crc.c]{An implementation of checksum algorithms. (194 LOC)}
\end{description}

If you are using another programming language, you are free to re-use
an existing CRC implementation in that language.


The main file for the exercise is {\tt arp.c}. In this file, you
should implement a program {\tt arp} which answers to ARP requests,
can initiate ARP requests from the command line and cache the answers.


\section{Provided binaries} \label{sec:binaries}

You are provided with several binaries:
\begin{description}
\item[reference-test-arp] A public test case, run using ``./reference-test-arp ./arp''
  to test your ARP implementation. Returns 0 on success.
\item[reference-arp] Reference implementation of the ``arp''.
\item[bug1-arp] Buggy implementation of a ``arp``.
\item[bug2-arp] Buggy implementation of a ``arp``.
\end{description}

\newpage
\section{Required make targets}

You may modify the build system. However, the final build system must
have the following {\tt make} targets:

\begin{description}
\item[all] build all binaries
\item[clean] remove all compiled files
\item[arp] build your ``arp`` binary from source; the binary MUST end up in the top-level directory of your build tree.
\item[test-arp] build your ``test-arp`` program from source; the program MUST end up in the top-level directory of your build tree.
\item[check-arp] Run ``test-arp`` against the ``arp'' binary.
\end{description}

For grading, we will basically run commands like:
\begin{verbatim}
GRADE=0
make test-arp
cp bug1-arp arp
make check-arp || GRADE=`expr $GRADE + 2`
cp bug2-arp arp
make check-arp || GRADE=`expr $GRADE + 2`
cp reference-arp arp
make check-arp || GRADE=0
echo "Test grade: $GRADE"
\end{verbatim}

You must thus make sure the build system continues to create programs in the
right (top-level) location!


\end{document}
