\documentclass{article}
\usepackage{url}
\usepackage{upquote}

\title{BTI 3021: Networking Project - Repeaters}

\author{Christian Grothoff}
\date{}

\begin{document}
\maketitle

\section{Introduction}

This project is ONLY for students repeating BTI 3021!  Unlike the
three sprints for first-time students, you must work in teams of {\bf
  at most two students} for this project.  To pass the networking part
of BTI 3021, you must achieve a total of {\bf 24} points in this
project (for a team of two).  The deadline is the same deadline as for
sprint 3.

For this project you will implement, document and test an Ethernet
{\bf vswitch} in userland under GNU/Linux.

While the driver and skeleton you are given is written in C, you may
use {\em any} language of your choice for the implementation (as long
as you extend the {\tt Makefile} with adequate build rules).  However,
if you choose a different language, be prepared to write additional
boilerplate yourselves.

How an Ethernet virtual switch (vswitch) works is expected to be
understood from the networking class. If not, you can find plenty of
documentation and specifications on the Internet.


\subsection{Deliverables}

There will be two main deliverables for the sprint:

\begin{description}
\item[Implementation] You must {\bf implement the switching
  algorithm}, extending the {\tt vswitch.c} template provided
  (or write the entire logic from scratch in another language).
\item[Testing] You must implement and submit your own {\bf test cases}
  by {\em pretending} to be the network driver (see below) and sending
  various Ethernet frames (with and without VLAN tags)
  to your program and verifying that it
  outputs the correct frames. Additionally, you should perform
  {\em interoperability} tests against existing
  implementations (i.e. other notebooks from your team to ensure that
  your switch integrates correctly with other implementations).
\end{description}


\subsection{Functionality}

Implement {\em vswitch} which forwards frames received on any
interface to any other interface, passively learns MAC addresses,
and respects VLAN tags. As before, the command-line specifies the
list of network interfaces you should switch on, but with
additional options to specify the VLANS.  Example:
\begin{verbatim}
$ network-driver eth0 eth1 eth2 eth3 - \
  vswitch eth0[T:1,2] eth1[U:1] eth2[U:2] eth3[U:2]
\end{verbatim}
This is supposed to run VLANs 1 and 2 tagged on {\tt eth0},
and VLANs 1, 2 or 2 untagged on {\tt eth1}, {\tt eth2},
or {\tt eth3} respectively.  Network interfaces specified
without ``[]'' should operate untagged on VLAN 0.  It is
not allowed to have interfaces accept both tagged and
untagged frames.

You may want to test your implementation against the Netgear
switch of the lab. For example, you could
Bridge a tagged VLAN ({\tt VID}$=3$) from the Netgear switch ({\tt
  eth1}) with two untagged notebooks ({\tt eth2}, {\tt eth3}) using
your vswitch code.


We will specifically also look for the following properties of a vswitch:
\begin{itemize}
\item Adding and stripping VLAN tags
\item Proper separation of VLANs
\item Support for multicast and broadcast
\item Changing external connections (re-learning when devices move around the network)
\item Managing an ``attacker'' process that sends from billions of MAC
  addresses.  Ensure your vswitch's learning table uses finite memory.
  You may use a single global learning table or separate tables per VLAN.
\end{itemize}

\subsection{Testing}

Note that in addition to automated tests similar to the {\tt
  public-test-router} you {\bf should} probably do integration tests
in a real network (evaluating manually with ping, wireshark, etc.).
This will help you find problems in your implementation and ensure
that your understanding of the network protocols is correct.


\section{Grading}

{\bf All deliverables must be submitted to Git} (master branch) by the
submission deadline announced on Moodle.

You are expected to work in a team of {\bf four students}.  If needed,
the course coordinator may permit the creation of a team of 5
students.  Each team is responsible for dividing up the work and
coordinating as needed.

\subsection{BTI 3021}
You can earn 14, 14 and 22 points in the three networking sprints, for
a total of 50 points.  A canonical team will need {\bf 37 points} to
pass the networking component of BTI 3021.

\subsection{BTI 3022}
You can earn 14 and 22 points in the three networking sprints, for
a total of 36 points.  A canonical team will need {\bf 27 points} to
pass the networking component of BTI 3022.

\subsection{BTI 302x repeaters}

You can earn 24 points in a single sprint.
A team of two will need all {\bf 24 points}
pass the networking component of BTI 302x
when repeating.
A team of one will need {\bf 20 points} to
pass the networking component of BTI 302x
when repeating.

\subsection{Smaller teams for non-repeaters}

If your team is {\bf for good reasons} smaller than four students, the
passing threshold will be lowered by {\bf 2 points} per ``missing''
student per sprint.  So a student going alone in BTI 3021 would still
need {\bf 19 points} to pass.

Teams may {\bf request} changes to team membership at the end of a
sprint, but must provide a {\bf justification} to the course
coordinator, who may approve or decline the request.


\subsection{VSwitch grading}

For the {\bf vswitch} sprint, you get points for each of the key deliverables:
\begin{center}
\begin{tabular}{l|r}
Correct implementation                    & 18 \\ \hline
Comprehensive test cases                  &  6 \\ \hline \hline
Total                                     & 24
\end{tabular}
\end{center}

\subsubsection{Correct implementation}
\begin{itemize}
  \item 18 points for passing test cases
\end{itemize}

\subsubsection{Comprehensive test cases}
\begin{itemize}
\item 0 points if public reference implementation (see Section~\ref{sec:binaries})
      fails test cases, {\bf otherwise}
\item 6 points for failing buggy implementations (see Section~\ref{sec:binaries})
\end{itemize}

If you believe you have found a bug in the provided reference
implementation, you are encouraged to discuss it with the
instructor. If you have found an actual bug in the reference
implementation (that is within the scope of the assignment), you will
be awarded a {\bf bonus point} per acknowledged bug.


\section{Hardware}

You will be given a 4-port Ethernet USB adapter that you can use to
add four physical ports to any PC.  If you use the laboratory PCs, be
aware that some of the USB ports provide insufficient power. Which
ones work is inconsistent even across identical PCs and often even the
adjacent USB port works even though it looks identical!

You are not expected to write a driver to interact directly with the
Ethernet USB adapter.  Instead, you will use the provided
{\tt network-driver} which can already provide you with raw access
to any Ethernet interface (incl. WLAN).

\subsection{Alternative setup with virtual machines}


Clone the Git repository at
\url{https://gitlab.ti.bfh.ch/demos/vlab} and follow the provided
instructions.

\section{The {\tt network-driver}}

To access the hardware, your final program should be {\em executed}
by the {\tt network-driver}.  For this, you call
\begin{verbatim}
$ network-driver IFC1 ... IFCn - ./switch ARGS
\end{verbatim}
where ``IFC1 ... IFCn'' is the list of interface names that you want
{\tt network-driver} to support (i.e. ``lan0'', ``lan1'') and ``PROG''
is the name of your binary and ``ARGS'' are the command-line arguments
to ``switch''.  Note the ``-'' (single minus) between the last
interface name and ``switch''.  Also, ``./switch'' must be given with
its path (i.e. ``./switch'' for the current working directory) or be
located in a directory that is given in the ``PATH'' environment
variable.

Once you start {\tt switch} like this, you can read Ethernet frames
and end-user commands from ``stdin'' and write Ethernet
frames (and end-user output) to ``stdout''.

Note that you must follow the {\tt network-driver}'s particular
format for inter-process communication when reading and writing.
You will {\bf not} be communicating directly with the console!


\subsection{Build the driver}

To compile the code, run:
\begin{verbatim}
# This requires gcc
$ make
# Creating network interfaces requires 'root' rights
$ sudo chmod +s network-driver
# Try it out:
$ ./network-driver eth0 - ./parser
\end{verbatim}
Press CTRL-C to stop the {\tt network-driver} and {\tt parser}.


\subsection{Understanding the driver}

The output of the driver is always in binary and generally in network
byte order.  You can use a tool like {\tt hexer} to make the output
slightly more readable.

The driver will always output a series of messages starting with
a {\tt struct GLAB\_MessageHeader} that includes a type and a size.

When the driver starts, it first writes a control message (of type 0)
with payload that includes 6 bytes for each of the local interface's
MAC addresses to your {\tt stdin}.  Henceforce, messages received
of type 0 will be single lines of command-line input (including the
'\\n'-terminator, but excluding the 0-terminator of C) as typed in
by the user.

Furthermore, the driver will output a {\tt struct GLAB\_MessageHeader}
for each frame received.  The {\tt struct GLAB\_MessageHeader} will be
followed by the actual network frame, starting with the Ethernet frame
excluding preamble, delimiter and FCS.  The {\tt struct
  GLAB\_MessageHeader} includes the total length of the subsequent
frame (encoded in network byte order, the size includes the {\tt
  struct GLAB\_MessageHeader}).  The fixed message type identifies the
number of the network interface, counting from one (also in network
byte order).

In addition to writing received frames to your {\tt stdin}, the driver
also tries to read from your {\tt stdout}.  Applications must send the
same message format to {\tt stdout} that the driver sends them on {\tt
  stdin}.  The driver does {\bf not} check that the source MAC is set
correctly!

To write to the console's {\tt stdout}, use a message type of 0.
You may directly write to {\tt stderr} for error messages.


\section{Provided code}


You are given a few C snippets as starting points. However, these
mostly serve to {\em illustrate} how to process the output from the
driver. You are completely free to implement your application in {\em
  any} programming language.  Note that each file includes about 20
LOC of a licensing statement, so the functions provided should not
provide a significant advantage for implementations in C.

\begin{description}
\item[parser.c]{This file includes a simple starting point for
  a wireshark-like frame inspection code.
  It mostly shows how the frames are received and
  a bit how to use the other C files. (82 LOC)}
\item[glab.h]{A struct defining a MAC Address and a few common C includes. (90 LOC)}
\item[print.c]{This file shows how to wrap messages to print them
  via the driver. (112 LOC)}
\item[loop.c]{This could be the main loop of your application. The code includes
  some basic logic that looks at each frame, decides whether it is the MACs,
  control or an Ethernet frame and then calls the respective function. (93 LOC)}
\item[crc.c]{An implementation of checksum algorithms. (194 LOC)}
\end{description}

If you are using another programming language, you are free to re-use
an existing CRC implementation in that language.


The main file for the exercise is {\tt vswitch.c}. In this file, you should
implement a program {\tt vswitch} which forwards frames received on any
interface to any other interface, but passively learns MAC addresses
and optimizes subsequent traffic.


\section{Provided binaries} \label{sec:binaries}

You are provided with several binaries:

\begin{description}
\item[reference-test-vswitch] A public test case, run using ``./reference-test-vswitch ./vswitch''
  to test your vswitch. Returns 0 on success.
\item[reference-vswitch] Reference implementation of the ``vswitch''.
\item[bug1-vswitch] Buggy implementation of a ``vswitch''.
\item[bug2-vswitch] Buggy implementation of a ``vswitch''.
\item[bug3-vswitch] Buggy implementation of a ``vswitch''.
\end{description}


\section{Required make targets}

You may modify the build system. However, the final build system must
have the following {\tt make} targets:

\begin{description}
\item[all] build all binaries
\item[clean] remove all compiled files
\item[vswitch] build your ``vswitch`` binary from source; the binary MUST end up in the top-level directory of your build tree.
\item[test-vswitch] build your ``test-vswitch`` program from source; the program MUST end up in the top-level directory of your build tree.
\item[check-vswitch] Run ``test-vswitch`` against the ``vswitch'' binary.
\end{description}

For grading, we will basically run commands like:
\begin{verbatim}
GRADE=0
make test-vswitch
cp bug1-vswitch vswitch
make check-vswitch || GRADE=`expr $GRADE + 2`
cp bug2-vswitch vswitch
make check-vswitch || GRADE=`expr $GRADE + 2`
cp bug3-vswitch vswitch
make check-vswitch || GRADE=`expr $GRADE + 2`
cp reference-vswitch vswitch
make check-vswitch || GRADE=0
echo "Test grade: $GRADE"
\end{verbatim}

You must thus make sure the build system continues to create programs
in the right (top-level) location!

\end{document}
